\documentclass{ctexbook}

\usepackage{xeCJK}
\usepackage{amsmath}
\usepackage{amsthm}
\usepackage{amssymb}
\usepackage{amsfonts}
\usepackage{bm}
\usepackage{bbm}

\theoremstyle{definition}
\newtheorem{definition}{Definition}[section]
\newtheorem{proposition}[definition]{Proposition}
\newtheorem{lemma}[definition]{Lemma}
\newtheorem{theorem}[definition]{Theorem}
\newtheorem{example}[definition]{Example}
\theoremstyle{remark}
\newtheorem{plain}[definition]{Remark}

\newcommand{\dif}{\mathrm{d}}

\title{非线性Schr\"odinger方程笔记}
\author{dxww}

\begin{document}
\maketitle
\tableofcontents

\chapter{分析工具箱}
\section{$\mathbb{R}^d$上的Fourier分析}
\section{$\mathbb{T}^d$上的Fourier分析}
在$\mathbb{T}^d$上做Fourier变换之前, 我们首先需要澄清什么是$\mathbb{T}^d$. 作为一个加法群, 我们定义$\mathbb{T}^d=\mathbb{R}^d/(2\pi\mathbb{Z}^d)$, 注意$2\pi\mathbb{Z}^d$是$\mathbb{R}^d$的子群. 按这个定义可知, $\mathbb{T}^d$可以看作$[-\pi,\pi]^d$把对边粘起来, 事实上群的商映射$p:\mathbb{R}^d\rightarrow\mathbb{T}^d$同时也是拓扑上的商映射. 

接着我们需要定义$\mathbb{T}^d$上的测度.取$p':[-\pi,\pi]^d\rightarrow\mathbb{T}^d$是粘合映射, 我们定义$E\subset\mathbb{T}^d$的测度为$p'^{-1}(E)$在$[-\pi,\pi]$上的测度. 这个测度也是$\mathbb{T}^d$上的Harr测度. 我们不用特别的记号去标记这个测度, 因为它和$\mathbb{R}^d$上的Lebesgue测度限制到$[-\pi,\pi]^d$上是完全一样的. 

$\mathbb{T}^d$上的任何函数$f$也可以看作$\mathbb{R}^d$上的周期函数$\tilde{f}=f\circ p$, 显然
$$\int_{\mathbb{T}^d}f\dif x=\int_{[-\pi,\pi]^d}\tilde{f}\dif x$$
以后我们都这么计算$f$的积分. 我们在谈论一个$\mathbb{R}^d$上的函数的时候, 也可以认为是在谈论$\mathbb{T}^d$上的一个周期函数, 譬如说, 我们会说$f(x)=e^{i\xi\cdot x}\ (\xi\in\mathbb{Z}^d)$是$\mathbb{T}^d$上的函数. 

\begin{definition}
设$f\in L^1(\mathbb{T}^d)$, 定义$f$的Fourier变换$\widehat{f}:\mathbb{Z}^d\rightarrow\mathbb{C}$为
$$\widehat{f}(\xi)=\frac{1}{(2\pi)^d}\int_{\mathbb{T}^d}f(x)e^{ix\cdot\xi}\dif x$$
\end{definition}
\section{Bochner积分}
\section{Sobolev空间}
\section{$U^p$和$V^p$}
\section{$X^s$和$Y^s$}

\chapter{$\mathbb{R}^d$上的非线性Schr\"odinger方程}
我们要考虑的方程是
$$iu_t+\Delta u=\mu|u|^{p-1}u$$
这里$p>1$, $\mu=\pm1$. 当$\mu=-1$时, 我们说这个方程是散焦型(defocusing)的, $\mu=1$时称为聚焦型(focusing). 
\section{线性方程}
\section{Strichartz估计}
\section{局部理论}
\section{守恒律}
\section{整体理论}
\section{散射}
\section{基态}


\end{document}